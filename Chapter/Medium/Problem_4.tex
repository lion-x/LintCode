\section{Problem ID: 4 Ugly Number II}
\subsection{Description}
Ugly number is a number that only have factors 2, 3 and 5.

Design an algorithm to find the nth ugly number. The first 10 ugly numbers are 1, 2, 3, 4, 5, 6, 8, 9, 10, 12...

\textbf{Notice}
Note that 1 is typically treated as an ugly number.

\subsection{Example}
If n=9, return 10.

\subsection{Code}
\scriptsize
\subsubsection{C++}
\begin{lstlisting}[language=C++]
class Solution {
public:
    /*
     * @param n an integer
     * @return the nth prime number as description.
     */
    int nthUglyNumber(int n) {
        // write your code here
        int *uglys = new int[n];
        uglys[0] = 1;
        int next = 1;
        int *p2 = uglys;
        int *p3 = uglys;
        int *p5 = uglys;
        while(next < n){
            int m = min(min(*p2 * 2, *p3 * 3), *p5 *5);
            uglys[next] = m;
            while(*p2 * 2 <= uglys[next]){
                p2++;
            }
            while(*p3 * 3 <= uglys[next]){
                p3++;
            }
            while(*p5 * 5 <= uglys[next]){
                p5++;
            }
            next++;
        }
        int uglyNum = uglys[n - 1];
        delete[] uglys;
        return uglyNum;
    }
};
\end{lstlisting}

\subsubsection{Python}
\begin{lstlisting}[language=Python]
class Solution:
    """
    @param {int} n an integer.
    @return {int} the nth prime number as description.
    """
    def nthUglyNumber(self, n):
        # write your code here
        uglys = []
        uglys.append(1)
        p2 = 0
        p3 = 0
        p5 = 0
        next = 1
        while next < n:
            m = min(uglys[p2] * 2, uglys[p3] * 3, uglys[p5] * 5)
            uglys.append(m)
            while uglys[p2] * 2 <= uglys[next]:
                p2 = p2 + 1
            while uglys[p3] * 3 <= uglys[next]:
                p3 = p3 + 1
            while uglys[p5] * 5 <= uglys[next]:
                p5 = p5 + 1
            next = next + 1
        return uglys[n - 1]
\end{lstlisting}
\normalsize 