\section{Problem ID: 428 Pow(x, n)}
\subsection{Description}
Implement pow(x, n).

\subsection{Example}
Pow(2.1, 3) = 9.261

Pow(0, 1) = 0

Pow(1, 0) = 1

\subsection{Code}
\scriptsize
\subsubsection{C++}
\begin{lstlisting}[language=C++]
class Solution {
public:
    /**
     * @param x the base number
     * @param n the power number
     * @return the result
     */
    double myPow(double x, int n) {
        // Write your code here
        if(n < 0){
            return 1.0 / myPow(x, -n);
        }
        if(n == 0){
            return 1;
        }
        if(n % 2 == 0){
            return  myPow(x, n>>1) * myPow(x, n>>1);
        }else{
            return  myPow(x, n>>1) * myPow(x, n>>1) * x;
        }
    }
};
\end{lstlisting}

\subsubsection{Python}
\begin{lstlisting}[language=Python]
class Solution:
    # @param {double} x the base number
    # @param {int} n the power number
    # @return {double} the result
    def myPow(self, x, n):
        # Write your code here
        if n < 0:
            return 1.0 / self.myPow(x, -n)
        if n == 0:
            return 1
        if n % 2 == 0:
            return self.myPow(x, n >> 1) * self.myPow(x, n >> 1)
        else:
            return self.myPow(x, n >> 1) * self.myPow(x, n >> 1) * x
\end{lstlisting}
\normalsize 