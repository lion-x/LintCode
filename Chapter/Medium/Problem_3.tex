\section{Problem ID: 3 Digit Counts}
\subsection{Description}
Count the number of k's between 0 and n. k can be 0 - 9.

\subsection{Example}
if n=12, k=1 in [0, 1, 2, 3, 4, 5, 6, 7, 8, 9, 10, 11, 12], we have FIVE 1's (1, 10, 11, 12)

\subsection{Code}
\scriptsize
\subsubsection{C++}
\begin{lstlisting}[language=C++]
class Solution {
public:
    /*
     * param k : As description.
     * param n : As description.
     * return: How many k's between 0 and n.
     */
    int digitCounts(int k, int n) {
        // write your code here
        int count = 0;
        for(int i = 0; i <= n; i++){
            int j = i;
            while(true){
                if(j % 10 == k){
                    count++;
                }
                j = j / 10;
                if(j == 0){
                    break;
                }
            }
        }
        return count;
    }
};
\end{lstlisting}

\subsubsection{Python}
\begin{lstlisting}[language=Python]
class Solution:
    # @param k & n  two integer
    # @return ans a integer
    def digitCounts(self, k, n):
        assert(n >= 0 and 0 <= k <= 9)
        count = 0
        for i in range(n + 1):
            j = i
            while True:
                if j % 10 == k:
                    count = count + 1
                j = j // 10
                if j == 0:
                    break
        return count
\end{lstlisting}
\normalsize 