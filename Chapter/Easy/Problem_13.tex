\section{Problem ID: 13 strStr}
\subsection{Description}
For a given source string and a target string, you should output the first index(from 0) of target string in source string.

If target does not exist in source, just return -1.

\subsection{Clarification}
Do I need to implement KMP Algorithm in a real interview?

Not necessary. When you meet this problem in a real interview, the interviewer may just want to test your basic implementation ability. But make sure your confirm with the interviewer first.

\subsection{Example}
If source = "source" and target = "target", return -1.

If source = "abcdabcdefg" and target = "bcd", return 1.

\subsection{Code}
\scriptsize
\subsubsection{C++}
\begin{lstlisting}[language=C++]
#include <cstring>
#include <iostream>
using namespace std;
class Solution {
public:
    /**
     * Returns a index to the first occurrence of target in source,
     * or -1  if target is not part of source.
     * @param source string to be scanned.
     * @param target string containing the sequence of characters to match.
     */
    int strStr(const char *source, const char *target) {
        // write your code here
        if (source == NULL || target == NULL){
            return -1;
        }
        int size_source = strlen(source);
        int size_target = strlen(target);
        int i, j;
        for (i = 0; i < size_source - size_target + 1; i++){
            for(j = 0; j <size_target; j++){
                if(source[i + j] != target[j]){
                    break;
                }
            }
            if(j == size_target){
                return i;
            }
        }
        return -1;
    }
};
\end{lstlisting}

\subsubsection{Python}
\begin{lstlisting}[language=Python]
class Solution:
    def strStr(self, source, target):
        # write your code here
        if source is None or target is None:
            return -1
        return source.find(target)
\end{lstlisting}
\normalsize 