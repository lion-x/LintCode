\section{Problem ID: 166 Nth to Last Node in List}
\subsection{Description}
Find the nth to last element of a singly linked list.

The minimum number of nodes in list is n.

\subsection{Example}
Given a List  3->2->1->5->null and n = 2, return node  whose value is 1.

\subsection{Code}
\scriptsize
\subsubsection{C++}
\begin{lstlisting}[language=C++]
/**
 * Definition of ListNode
 * class ListNode {
 * public:
 *     int val;
 *     ListNode *next;
 *     ListNode(int val) {
 *         this->val = val;
 *         this->next = NULL;
 *     }
 * }
 */
class Solution {
public:
    /**
     * @param head: The first node of linked list.
     * @param n: An integer.
     * @return: Nth to last node of a singly linked list.
     */
    ListNode *nthToLast(ListNode *head, int n) {
        // write your code here
        if(head == NULL || n < 1){
            return NULL;
        }
        ListNode *p1 = head;
        ListNode *p2 = head;
        for(int i = 0; i < n; i++){
            if(p2 != NULL){
                p2 = p2->next;
            }else{
                return NULL;
            }
        }
        while(p2 != NULL){
            p1 = p1->next;
            p2 = p2->next;
        }
        return p1;
    }
};
\end{lstlisting}

\subsubsection{Python}
\begin{lstlisting}[language=Python]
"""
Definition of ListNode
class ListNode(object):

    def __init__(self, val, next=None):
        self.val = val
        self.next = next
"""
class Solution:
    """
    @param head: The first node of linked list.
    @param n: An integer.
    @return: Nth to last node of a singly linked list.
    """
    def nthToLast(self, head, n):
        # write your code here
        if head is None or n < 1:
            return None
        p1 = head
        p2 = head
        for i in range(0,n):
            if p2 is not None:
                p2 = p2.next
            else:
                return None
        while p2 is not None:
            p2 = p2.next
            p1 = p1.next
        return p1
\end{lstlisting}
\normalsize 