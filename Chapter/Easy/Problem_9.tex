\section{Problem ID: 9  Fizz Buzz}
\subsection{Description}
Given number n. Print number from 1 to n. But:

when number is divided by 3, print "fizz".

when number is divided by 5, print "buzz".

when number is divided by both 3 and 5, print "fizz buzz".

\subsection{Example}
If n = 15, you should return:

[

  "1", "2", "fizz",
  
  "4", "buzz", "fizz",
  
  "7", "8", "fizz",
  
  "buzz", "11", "fizz",
  
  "13", "14", "fizz buzz"
  
]

\subsection{Code}
\scriptsize
\subsubsection{C++}
\begin{lstlisting}[language=C++]
class Solution {
public:
    /**
     * param n: As description.
     * return: A list of strings.
     */
    vector<string> fizzBuzz(int n) {
        vector<string> results;
        for (int i = 1; i <= n; i++) {
            if (i % 15 == 0) {
                results.push_back("fizz buzz");
            } else if (i % 5 == 0) {
                results.push_back("buzz");
            } else if (i % 3 == 0) {
                results.push_back("fizz");
            } else {
                results.push_back(to_string(i));
            }
        }
        return results;
    }
};
\end{lstlisting}

\subsubsection{Python}
\begin{lstlisting}[language=Python]
class Solution:
    """
    @param n: An integer as description
    @return: A list of strings.
    For example, if n = 7, your code should return
        ["1", "2", "fizz", "4", "buzz", "fizz", "7"]
    """
    def fizzBuzz(self, n):
        results = []
        for i in range(1, n+1):
            if i % 15 == 0:
                results.append("fizz buzz")
            elif i % 5 == 0:
                results.append("buzz")
            elif i % 3 == 0:
                results.append("fizz")
            else:
                results.append(str(i))
        return results
\end{lstlisting}
\normalsize 