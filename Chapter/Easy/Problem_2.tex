\section{Problem ID: 2 Trailing Zeros}
\subsection{Description}
Write an algorithm which computes the number of trailing zeros in n factorial.

\subsection{Example}
11! = 39916800, so the out should be 2

\subsection{Code}
\scriptsize
\subsubsection{C++}
\begin{lstlisting}[language=C++]
class Solution {
 public:
    // param n : description of n
    // return: description of return
    long long trailingZeros(long long n) {
        long long sum = 0;
        while(n!=0){
            sum += n / 5;
            n = n / 5;
        }
        return sum;
    }
};
\end{lstlisting}

\subsubsection{Python}
\begin{lstlisting}[language=Python]
class Solution:
    # @param n a integer
    # @return ans a integer
    def trailingZeros(self, n):
        sum = 0
        while n != 0:
            sum += n // 5
            n = n // 5
        return sum
\end{lstlisting}
\normalsize